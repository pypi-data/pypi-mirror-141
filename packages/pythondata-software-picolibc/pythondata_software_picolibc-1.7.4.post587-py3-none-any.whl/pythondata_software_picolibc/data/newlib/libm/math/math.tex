#
#Copyright (C) 1993 by Sun Microsystems, Inc. All rights reserved.
#
#Developed at SunPro, a Sun Microsystems, Inc. business.
#Permission to use, copy, modify, and distribute this
#software is freely granted, provided that this notice 
#is preserved.
#
@node Math
@chapter Mathematical Functions (@file{math.h})

This chapter groups a wide variety of mathematical functions.  The
corresponding definitions and declarations are in @file{math.h}.  
The definition of HUGE_VAL from @file{math.h} is of particular interest.

@enumerate
@item
The representation of infinity as a @code{double} is defined as
@code{HUGE_VAL}; this number is returned on overflow by many functions.
The macro @code{HUGE_VALF} is a corresponding value for @code{float}.
@end enumerate

Alternative declarations of the mathematical functions, which exploit
specific machine capabilities to operate faster---but generally have
less error checking and may reflect additional limitations on some
machines---are available when you include @file{fastmath.h} instead of
@file{math.h}.

@menu 
* version::	Version of library
* acos::	Arccosine
* acosh::	Inverse hyperbolic cosine
* asin::	Arcsine
* asinh::	Inverse hyperbolic sine
* atan::	Arctangent
* atan2::	Arctangent of y/x
* atanh::	Inverse hyperbolic tangent
* jN::	        Bessel functions (jN, yN)
* cbrt::	Cube root
* copysign::	Sign of Y, magnitude of X
* cosh::	Hyperbolic cosine
* erf::		Error function (erf, erfc)
* exp::		Exponential, base e
* exp10::	Exponential, base 10
* exp2::	Exponential, base 2
* expm1::	Exponential, base e, of x - 1
* fabs::	Absolute value (magnitude)
* fdim::	Positive difference
* floor::	Floor and ceiling (floor, ceil)
* fma::		Floating multiply add
* fmax::	Maximum
* fmin::	Minimum
* fmod::	Floating-point remainder (modulo)
* fpclassify::	Floating-point classification macro
* frexp::	Split floating-point number
* gamma::	Logarithmic gamma function
* hypot::	Distance from origin
* ilogb::	Get exponent
* infinity::	Floating infinity
* isgreater::	Comparison macros
* ldexp::	Scale by a power of 2
* log::		Natural logarithms
* log10::	Base 10 logarithms
* log1p::	Log of 1 + X
* log2::	Base 2 logarithms
* logb::	Get exponent
* lrint::	Round to integer
* lround::	Round to integer, away from zero (lround, llround)
* modf::	Split fractional and integer parts
* nan::		Floating Not a Number
* nearbyint::	Round to integer
* nextafter::	Get next representable number
* pow::		X to the power Y
* pow10::	10 to the power X
* remainder::	remainder of X divided by Y 
* remquo::	Remainder and part of quotient
* rint::	Round to integer
* round::	Round to integer, away from zero
* scalbn::	Scale by a power of FLT_RADIX (2)
* signbit::	Does floating-point number have negative sign?
* sin::		Sine or cosine (sin, cos)
* sinh::	Hyperbolic sine
* sqrt::	Positive square root
* tan::		Tangent
* tanh::	Hyperbolic tangent
* trunc::	Round to integer, towards zero
@end menu

@page
@node version
@section Error Handling

There are two different versions of the math library routines: IEEE
and POSIX.  The version may be selected at runtime by
setting the global variable @code{_LIB_VERSION}, defined in
@file{math.h}.  It may be set to one of the following constants defined
in @file{math.h}: @code{_IEEE_} or @code{_POSIX_}.
The @code{_LIB_VERSION} variable is not specific to any
thread, and changing it will affect all threads.

The versions of the library differ only in the setting of @code{errno}.

In IEEE mode, @code{errno} is never set.

In POSIX mode, @code{errno} is set correctly.

The library is set to IEEE mode by default.

The majority of the floating-point math functions are written
so as to produce the floating-point exceptions (e.g. "invalid",
"divide-by-zero") as required by the C and POSIX standards, for
floating-point implementations that support them.  Newlib does not provide
the floating-point exception access routines defined in the standards
for fenv.h, though, which is why they are considered unsupported.  It is
mentioned in case you have separately-provided access routines so that
you are aware that they can be caused.

@section Standards Compliance And Portability
Most of the individual function descriptions describe the standards to which
each function complies.  However, these descriptions are mostly out of date,
having been written before C99 was released.  One of these days we'll get
around to updating the rest of them.  (If you'd like to help, please let us
know.)

``C99'' refers to ISO/IEC 9899:1999, ``Programming languages--C''.
``POSIX'' refers to IEEE Standard 1003.1.  POSIX@registeredsymbol{} is a
registered trademark of The IEEE.

@c To sort the include list easily, keep the indentation right because want to
@c skip the s_|w_ at the start of most--but not all--of the file names.
@c (e.g., isgreater.def does not have a leading s nor w.)  Then, sort
@c based on the column.  For example:  "sort -t@ -k3.17"
@c A few hand-edits might be appropriate after a sort, although not necessary
@c and are a nuisance as ought to be kept in sync with menu list above:
@c atan2 after atan, exp2 after exp, log first in log list, and w_j0 to place
@c to reflect function name of Bessel (as opposed to j; e.g. after atanh,
@c before cbrt).

@page
@include   math/w_acos.def
@page
@include   math/w_acosh.def
@page
@include   math/w_asin.def
@page 
@include   math/s_asinh.def
@page
@include   math/s_atan.def
@page 
@include   math/w_atan2.def
@page
@include   math/w_atanh.def
@page 
@include   math/w_j0.def
@page
@include common/s_cbrt.def
@page
@include common/s_copysign.def
@page
@include   math/w_cosh.def
@page
@include   math/s_erf.def
@page
@include   math/w_exp.def
@page
@include   common/s_exp10.def
@page
@include   math/w_exp2.def
@page
@include common/s_expm1.def
@page
@include   math/s_fabs.def
@page
@include common/s_fdim.def
@page
@include   math/s_floor.def
@page
@include common/s_fma.def
@page
@include common/s_fmax.def
@page
@include common/s_fmin.def
@page
@include   math/w_fmod.def
@page
@include   math/s_frexp.def
@page
@include   math/w_gamma.def
@page
@include   math/w_hypot.def
@page
@include common/s_ilogb.def
@page
@include common/s_infinity.def
@page
@include   common/isgreater.def
@page
@include common/s_isnan.def
@page
@include   math/s_ldexp.def
@page
@include   math/w_log.def
@page
@include   math/w_log10.def
@page
@include common/s_log1p.def
@page
@include common/s_log2.def
@page
@include common/s_logb.def
@page
@include common/s_lrint.def
@page
@include common/s_lround.def
@page
@include common/s_modf.def
@page
@include common/s_nan.def
@page
@include common/s_nearbyint.def
@page
@include common/s_nextafter.def
@page
@include   math/w_pow.def
@page
@include   common/s_pow10.def
@page
@include   math/w_remainder.def
@page
@include common/s_remquo.def
@page
@include common/s_rint.def
@page
@include common/s_round.def
@page
@include common/s_scalbn.def
@page
@include common/s_signbit.def
@page
@include   math/s_sin.def
@page
@include   math/w_sinh.def
@page
@include   math/w_sqrt.def
@page
@include   math/s_tan.def
@page
@include   math/s_tanh.def
@page
@include common/s_trunc.def
