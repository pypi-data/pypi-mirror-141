%File: ~/OOP/system_of_eqn/linearSOE/FullGEN/FullGenLinSolver.tex
%What: "@(#) FullGenLinSolver.tex, revA"

\noindent {\bf Files}   \\
\#include $<\tilde{ }$/system\_of\_eqn/linearSOE/fullGEN/FullGenLinSolver.h$>$  


\noindent {\bf Class Declaration}  \\
class FullGenLinSolver: public LinearSOESolver  


\noindent {\bf Class Hierarchy} \\
MovableObject 

\indent\indent  Solver \\
\indent\indent\indent LinearSOESolver \\
\indent\indent\indent\indent {\bf FullGenLinSolver} \\

\noindent {\bf Description}  \\
\indent FullGenLinSolver is an abstract class.  The FullGEnLinSolver
class provides access for each subclass to the FullGenLinSOE object
through the pointer {\em theSOE}, which is a protected pointer. \\

\noindent {\bf Interface}  \\
\indent\indent // Constructor \\
\indent\indent {\em FullGenLinSolver(int classTag);}  \\ \\
\indent\indent // Destructor \\
\indent\indent {\em virtual~ $\tilde{}$FullGenLinSolver();}\\  \\
\indent\indent // Public Methods \\
\indent\indent {\em virtual int setLinearSOE(FullGenLinSOE \&theSOE);} \\

\noindent {\bf Constructor}  \\
{\em FullGenLinSolver(int classTag);}  

The integer {\em classTag} is passed to the LinearSOESolver classes
constructor. \\ 

\noindent {\bf Destructor} \\
\indent {\em virtual~ $\tilde{}$FullGenLinSolver();}\\ 
Does nothing, provided so the subclasses destructor will be called. \\

\noindent {\bf Public Methods }  \\
{\em virtual int setLinearSOE(FullGenLinSOE \&theSOE);} 

Sets the link to the FullGEnLinSOE object {\em theSOE}. This is the
object on which the solver will perform the numerical computations. \\




