%File ~/OOP/analysis/algorithm/ModifiedNewton.tex
%What: "@(#) ModifiedNewton.tex, revA"

\noindent {\bf Files}   \\
\indent \#include $<\tilde{
}$/analysis/algorithm/equiSolnAlgo/ModifiedNewton.h$>$  \\ 

\noindent {\bf Class Declaration}  \\
class ModifiedNewton: public EquiSolnAlg;  


\noindent {\bf Class Hierarchy} \\
MovableObject 

\indent\indent SolutionAlgorithm \\
\indent\indent\indent EquiSolnAlgo \\
\indent\indent\indent\indent {\bf ModifiedNewton} \\

\noindent {\bf Description} \\ 
\indent The ModifiedNewton class is an algorithmic class which obtains a
solution to a non-linear system using the modified Newton-Raphson iteration
scheme. The Newton-Rapson iteration scheme is based on a Taylor expansion of the
non-linear system of equations $\R(\U) = \zero$ about an approximate
solution $\U{(i)}$:
\begin{equation} 
\R(\U) = 
\R(\U^{(i)}) +
\left[ {\frac{\partial \R}{\partial \U} \vert}_{\U^{(i)}}\right]
\left( \U - \U^{(i)} \right) 
\end{equation}

\noindent which can be expressed as:
\begin{equation} \
\K^{(i)}  \Delta \U{(i)} = \R(\U^{(i)})
\end{equation}
which is solved for $\Delta \U^{(i)}$ to give approximation for
$\U^{(i+1)} = \U^{(i)} + \Delta \U^{(i)}$. To start the
iteration $\U^{(1)} = \U_{trial}$, i.e. the current trial
response quantities are chosen as initial response quantities. \\

in the modified version the tangent is formed only once, i.e
\begin{equation} \
\K^{(1)}  \Delta \U^{(i)} = \R(\U^{(i)})
\end{equation}

To stop the iteration, a test must be performed to see if convergence
has been achieved at each iteration. Each NewtonRaphson object is
associated with a ConvergenceTest object. It is this object which
determines if convergence has been achieved. \\

\noindent {\bf Class Interface} \\ 
\indent // Constructors \\ 
\indent {\em ModifiedNewton(ConvergenceTest \&theTest);}\\ 
{\em ModifiedNewton();}\\ 

// Destructor 

{\em ~ $\tilde{}$ModifiedNewton();}\\  

// Public Member Functions 

{\em int solveCurrentStep(void);} 

{\em void setTest(ConvergenceTest \&theTest);} \\

// Public Methods  for Output 

\indent {\em int sendSelf(int commitTag, Channel \&theChannel);}\\ 
\indent {\em int recvSelf(int commitTag, Channel \&theChannel, 
FEM\_ObjectBroker \&theBroker);}\\ 
{\em int Print(OPS_Stream \&s, int flag =0);} 



\noindent {\bf Constructors} \\ 
\indent {\em ModifiedNewton(int theMaxNumIter, ConvergenceTest \&theTest);}\\ 
The constructor takes as an argument the ConvregenceTest object {\em
theTest}, the object which is used at the end of each iteration to
determine if convergence has been obtained. The
integer {\em EquiALGORITHM\_TAGS\_ModifiedNewton} (defined in
$<$classTags.h$>$) is passed to the EquiSolnAlgo classes
constructor. \\ 

{\em ModifiedNewton();}

Provided for FEM\_ObjectBroker to instantiate a blank object with a
class tag of EquiALGORITHM\_TAGS\_ModofiedNewton. {\em
recvSelf()} must be invoked on this object. \\

\noindent {\bf Destructor} \\
\indent {\em ~ $\tilde{}$ModifiedNewton();}\\ 
Does nothing. \\

\noindent {\bf Public Member Functions}  \\
{\em int solveCurrentStep(void);} 

When invoked the object first sets itself as the EquiSolnAlgo object
that the ConvergenceTest is testing and then it performs the
modified Newton-Raphson iteration algorithm: 
\begin{tabbing}
while \= \+ while \= \kill
theTest-$>$start(); \\
theIntegrator-$>$formTangent(); \\
do \{ \+ \\
theIntegrator-$>$formUnbalance(); \\
theSOE-$>$solveX(); \\
theIntegrator-$>$update(theSOE-$>$getX()); \- \\
\} while (theTest-$>$test() $==$ false)\- \\
\end{tabbing}


\noindent The method returns a 0 if successful, otherwise a negative number is
returned; a $-1$ if error during {\em formTangent()}, a $-2$ if
error during {\em formUnbalance()}, a $-3$ if error during {\em
solve()}, and a $-4$ if error during {\em update()}.
If an error occurs in any of the above operations the method stops at
that routine, none of the subsequent operations are invoked. A $-5$ is
returned if any one of the links has not been setup.\\

{\em void setTest(ConvergenceTest \&theTest);} \\
A method to set the tolerance criteria of the algorithm to be equal to
the value {\em theTol}. \\

{\em int sendSelf(int commitTag, Channel \&theChannel);}\\
Creates an ID object, puts the values of the {\em theTest} objects
class and database tags into this ID.
It then invokes {\em sendVector()} on the Channel object {\em
theChannel} to send the data to the remote object. It then invokes
{\em sendSelf()} on {\em theTest}. Returns $0$ if successful, the
 channel error if not. \\

{\em int recvSelf(int commitTag, Channel \&theChannel, FEM\_ObjectBroker
\&theBroker);}\\ 
Creates an ID object, invokes {\em recvVector()} on the Channel
object. Uses the data in the ID to create a ConvergenceTest object of
appropriate type and sets its dbTag. It then invokes {\em recvSelf()}
on this test object.  \\

{\em int Print(OPS_Stream \&s, int flag =0);} \\
Sends the string 'ModifiedNewton' to the stream if {\em flag} equals $0$.
