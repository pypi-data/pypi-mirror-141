% File: ~/OOP/analysis/model/DOF_Group.tex 
%What: "@(#) DOF_Group.tex, revA"

NEED A GETTANGFORCE() LIKE FE\_ELEMENT FOR ELE\_BY\_ELE SOLVERS. KEEP A
POINTER TO LAST INTEGRATOR. \\

\noindent {\bf Files}   \\
\#include $<\tilde{ }$/analysis/dof\_grp/DOF\_Group.h$>$  


\noindent {\bf Class Declaration}  \\
class DOF\_Group;  


\noindent {\bf Description}  \\
\indent DOF\_Group is a base class. An object of type DOF\_Group
represents an unconstrained node of the domain in the model. Each node
in the domain is associated with one DOF\_Group. DOF\_Groups are
called upon in the analysis to provide their contributions of
unbalanced load to the system of equations. Subclasses are used by the
constraint handler to to introduce new dofs into the analysis. 

The DOF\_Group is responsible for providing operations to set and
access the mapping betwwen equation numbers and DOFs, allowing the
Integrator to from the tangent (if nodal masses in transient problem)
and unbalanced load information, and for setting and obtaining the
nodal trial response quantities. \\


\noindent {\bf Class Interface}  \\
// Constructors  

{\em DOF\_Group(int tag, Node *theNode);}  

{\em DOF\_Group(int tag, int numDOF);}  \\ 

// Destructor  

{\em virtual~ $\tilde{}$DOF\_Group();}  \\

// Public Methods - Mapping  

{\em virtual void setID(int dof, int value);} 

{\em virtual void setID(const ID \&values);} 

{\em virtual const ID \&getID(void) const;} 

{\em virtual int getTag(void) const;} 

{\em virtual int getNumDOF(void) const;}

{\em virtual int getNumFreeDOF(void) const;}

{\em virtual int getNumConstrainedDOF(void) const;}\\ 

\indent // Public Methods - Tangent \& Residual  \\ 
{\em virtual const Matrix \&getTangent(Integrator *theIntegrator)} 

{\em virtual void zeroTangent(void);}

{\em virtual void addMtoTang(double factt);}

\indent {\em virtual const Vector \&getUnbalance(Integrator
*theIntegrator); }\\ 
{\em virtual void zeroUnbalance(void);}

{\em virtual void  addPtoUnbalance(double fact = 1.0);} \\ 

\indent {\em virtual void addMtoTang(const Vector \&$\ddot u$,
double fact);}\\ \\ 
// Public Methods - Node Response  

{\em virtual const Vector \&getCommittedDisp(void);} 

{\em virtual const Vector \&getCommittedVel(void);} 

{\em virtual const Vector \&getCommittedAccel(void);} 

{\em virtual int setNodeDisp(const Vector \&u);} 

{\em virtual int setNodeVel(const Vector \&$\dot u$);}

{\em virtual int setNodeAccel(const Vector \&$\ddot u$);}

{\em virtual int incrNodeDisp(const Vector \&u);} 

{\em virtual int incrNodeVel(const Vector \&$\dot u$);}

{\em virtual int incrNodeAccel(const Vector \&$\ddot u$);}




\noindent {\bf Constructors}  \\
{\em DOF\_Group(int tag, Node *theNode);}  

Constructs a  DOF\_Group with an associated node given by {\em
theNode} and a tag given by {\em tag}. During construction it
determines the number of unknown dofs from the node, {\em numDOF}, and
creates an Vector object to hold the unbalance information and an ID
object to hold mapping between degrees-of-freedom and equation
numbers. All values of the ID are set to $-2$ initially. It creates a
Vector to hold the unbalance loads and an ID to 
hold the mapping information, both of size equal to the number of
unknowns. If the size of the Vector or the ID is $0$, i.e. ran out of
memory, a warning message is orinted and {\em numDOF} set to $0$. \\

{\em DOF\_Group(int tag, int numDOF);}  \\
Provided for subclasses. Constructs a  DOF\_Group with the number of
unknown dofs given by {\em numDOF} and a tag given by {\em tag}. No
Node is associated with this DOF\_Group. Creates a Vector of size {\em
numDOF} to hold the unbalance information and an ID object to
hold mapping between degrees-of-freedom and equation numbers. All
values of the ID are set to $-2$ initially. If
the size of the Vector or the ID is $0$, i.e. ran out of memory, a
warning message is orinted and {\em numDOF} set to $0$. \\

\noindent {\bf Destructor}  \\
{\em virtual~ $\tilde{}$DOF\_Group();}  

Invokes the destructor on the Vector created to hold the unbalance. \\

\noindent {\bf Public Methods - Equation Numbers}  \\
{\em virtual void setID(int dof, int value);} 

Operation to set the equation number of the {\em dof'th} DOF in the
DOF\_Group to {\em value}. In this base class, if {\em index} is a
valid location, $0 < index < numDOF-1$ the operator {\em (index) =
value} is invoked on the ID object created for the mapping, otherwise
an error message is printed. \\ 


{\em virtual void setID(const ID \&values);} \\
An operation to set the equation numbers of all the DOFs in the
DOF\_Group to be those given in the ID {\em values}. In this base
class, the operator {\em = values} is invoked on the ID object created
for the mapping. \\ 

{\em virtual const ID \&getID(void) const;} \\
A const member function which returns in an ID object the equation
numbers associated with the degrees-of-freedom in the DOF\_Group. The
size of the ID object is equal to the number of degrees-of-freedom
represented by the DOF\_Group. This base class returns the ID created
for the mapping.\\ 

{\em virtual int getTag(void) const;} \\
A const member function which returns the integer identifier {\em
tag} passed in the constructor. \\

{\em virtual int getNumDOF(void) const;}\\
Returns the total number of DOFs in the DOF\_Group. This base class
returns {\em numDOF}.\\

{\em virtual int getNumFreeDOF(void) const;}\\
Returns the total number of {\em free} DOFs in the DOF\_Group,
i.e. number of dof in the group who have not been assigned a negative
equation number. Determines this by looping through the ID containing
the mapping information. \\

{\em virtual int getNumConstrainedDOF(void) const;}\\
Returns the total number of {\em constrained} DOFs in the DOF\_Group,
i.e. number of dof in the group who have been assigned a negative
equation number. Determines this by looping through the ID containing
the mapping information. \\


\noindent {\bf Public Methods - Nodal Tangent contributions
for transient}  \\ 
{\em virtual const Matrix \&getTangent(void)} 

Returns the current tangent matrix for the DOF\_Group. If no tangent
matrix has been created, one of size(numDOF,numDOF) is constructed.
If not enough space is available for a new tangent matrix, an error
message is printed and an error Matrix is returned.. \\  

{\em virtual void formTangent(Integrator *theIntegrator)} 

To form the DOF\_Groups tangent. Invokes {\em formNodTangent(this)} on
The Integrator object {\em theIntegrator}. \\

{\em virtual void zeroTangent(void);}

To zero the tangent. If a tangent matrix has been allocated,
will invoke {\em Zero()} on the tangent matrix.\\

{\em virtual void addMtoTang(double fcat);}

To add {\em fact} times the nodal mass matrix to the tangent
matrix. In this base class, if a Matrix to store the tangent has not
yet been created, one of size (numDOF,numDOF) is now created; if
construction of this matrix fails an error is printed and an error
Matrix is returned. Invokes {\em addMatrix(theNode-$>$getMass,fact)}
on the Matrix, printing a warining message if this method returns a
$negative$ value. If no Node is associated with the DOF\_Group an
error message is printed and nothing is done. \\

\noindent {\bf Public Methods - Node Unbalance}  \\
{\em virtual void zeroUnbalance(void);}

To zero the unbalance vector. Invokes {\em Zero()} on the vector
object used to store the unbalance information. \\

{\em virtual void formUnbalance(Integrator *theIntegrator); }\\
Causes the DOF\_Group to form its contribution to the residual. Invokes
{\em formNodUnbalance(this)} on {\em theIntegrator}.\\

{\em virtual const Vector \&getUnbalance() const; } \\
Returns the vector holding the unbalance. \\

{\em virtual void  addPtoUnbalance(double fact = 1.0);} \\
Adds the product of the unbalanced load at the node and {\em fact} to
the unbalance vector. A warning message is printed and nothing is done
if no node is associated with the DOF\_Group, otherwise {\em
addVector(theNode-$>$getUnbalancedLoad(),fact)} is invoked on the
unbalance vector and a warning message is printed if this method
returns a negative value. \\ 

{\em virtual void addMtoTang(const Vector \&$\ddot u$,
double fact);}\\ \\ 
Adds {\em fact} times the product of the associated nodes mass matrix
and the Vector $\ddot u$ to the unbalance vector. A warning message is
printed and nothing is done if no node is associated with the
DOF\_Group. \\

\noindent {\bf Public Methods - Node Response}  \\
{\em virtual const Vector \&getCommittedDisp(void);} 

To return the committed displacement at the node. Returns the result
of invoking {\em getDisp()} on the Node. If there is no associated
node object, an error message is printed and an error Vector is
returned. \\


{\em virtual const Vector \&getCommittedVel(void);} 

To return the committed velocity at the node. Returns the result of
invoking {\em getVel()} on the Node. If there is no associated node
object, an error message is printed and an error Vector is returned. \\

{\em virtual const Vector \&getCommittedAccel(void);} 

To return the committed velocity at the node. Returns result of
invoking  {\em getAccel()} on the Node. If there is no associated node
object, an error message is printed and an error Vector is returned. \\

{\em virtual int setNodeDisp(const Vector \&u);} 

This operation sets the value of the nodal trial displacement at the
associated node. The vector {\em u} is of size equal to the number of
equations in the AnalysisModel (this is not checked by the DOF\_Group).
The DOF\_Group object accesses the contents of the Vector {\em u} using
its assigned equation numbers. If a $-1$ exists for a DOF a $0$ value
is set as the corresponding nodal displacement for the node. Creates a Vector
to store the correct components of {\em u}, then invokes {\em setTrialDisp()}
on the node object; if no node object an error message is
printed. CURRENTLY THERE IS NO CHECK TO SEE IF myID(i) DOES NOT OVERFLOW
ADMISSIBLE VALUES IN U - THIS NEEDS TO CHANGE \\


{\em virtual int setNodeVel(const Vector \&$u^{.}$);}\\
This operation sets the value of the nodal trial velocity at the
associated node. The vector {\em $u^{.}$} is of size equal to the number of
equations in the AnalysisModel (this is not checked by the DOF\_Group).
The DOF\_Group object accesses the contents of the Vector {\em $u^{.}$} using
its assigned equation numbers. If a $-1$ exists for a DOF a $0$ value
is set as the corresponding nodal velocity for the node.  Creates a Vector
to store the correct components of {\em u}, then invokes {\em setTrialVel()}
on the node object; if no node object an error message is printed. 
MUST CHANGE AS setNodeDisp \\

{\em virtual int setNodeAccel(const Vector \&$u^{..}$);}\\
This operation sets the value of the nodal trial acceleration at the
associated node. The vector {\em $u^{..}$} is of size equal to the number of
equations in the AnalysisModel (this is not checked by the DOF\_Group).
The DOF\_Group object accesses the contents of the Vector {\em $u^{..}$} using
its assigned equation numbers. If a $-1$ exists for a DOF a $0$ value
is set as the corresponding nodal acceleration for the node. The return
value is as outlined above for {\em setNodeAccel().} Creates a Vector
to store the correct components of {\em u}, then invokes {\em setTrialAccel()}
on the node object; if no node object an error message is printed. 
MUST CHANGE AS setNodeDisp \\



{\em virtual int incrNodeDisp(const Vector \&u);} 

{\em virtual int incrNodeVel(const Vector \&$u^{.}$);}

{\em virtual int incrNodeAccel(const Vector \&$u^{..}$);}

These methods are similar to those three just outlined, the only
difference being that the trial response quantities at the nodes are
incremented, not set, with the values now given; this is done by
invoking {\em incrTrialDisp()}, {\em incrTrialVel()}, and {\em 
incrTrialAccel()} respectively on the nodes. MUST ALL CHANGE AS
setNodeDisp \\ 






