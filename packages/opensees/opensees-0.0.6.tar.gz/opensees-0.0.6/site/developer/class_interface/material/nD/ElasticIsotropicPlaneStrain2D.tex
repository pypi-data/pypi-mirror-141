%File: ~/OOP/material/nD/ElasticIsotropicPlaneStrain2D.tex
%What: "@(#) ElasticIsotropicPlaneStrain2D.tex, revA"

\noindent {\bf Files}   \\
\#include $<\tilde{ }$/material/nD/ElasticIsotropicPlaneStrain2D.h$>$  


\noindent {\bf Class Declaration}  \\
class ElasticIsotropicPlaneStrain2D : public ElasticIsotropicMaterial 


\noindent {\bf Class Hierarchy} \\
TaggedObject 

MovableObject 

\indent\indent Material \\
\indent\indent\indent NDMaterial \\
\indent\indent\indent\indent ElasticIsotropicMaterial \\
\indent\indent\indent\indent\indent {\bf ElasticIsotropicPlaneStrain2D} \\

\noindent {\bf Description}  \\
ElasticIsotropicPlaneStrain2D provides the implementation of an
elastic isotropic material which exhibits plane strain behavior in two
dimensions. \\

\noindent {\bf Class Interface} \\
// Constructor 

\indent {\em ElasticIsotropicPlaneStrain2D (int tag, double E,
double v);}  \\ \\
// Destructor 

{\em $\tilde{ }$ElasticIsotropicPlaneStrain2D ();}\\ 

// Public Methods 

{\em int setTrialStrain (const Vector \&strain); } 

{\em const Vector \&getStress (void); } 

{\em const Matrix \&getTangent (void); } 

{\em int commitState (void); } 

{\em int revertToLastCommit (void); } 

{\em int revertToStart (void); } 

{\em NDMaterial *getCopy (void); } 


\noindent {\bf Constructor}  \\
{\em ElasticIsotropicPlaneStrain2D (int tag, double E, double v);}  

To construct an ElasticIsotropicPlaneStrain2D whose unique integer tag
among NDMaterials in the domain is given by {\em tag}.  The material
model have Young's modulus {\em E} and Poisson's ratio {\em v}. \\

\noindent {\bf Destructor} \\
{\em $\tilde{ }$ElasticIsotropicPlaneStrain2D ();} 

Does nothing. \\ 

\noindent {\bf Public Methods} \\
{\em int setTrialStrain (const Vector \&strain); }  

Sets the value of the current trial strain vector, $\myepsilon$,
to be {\em strain}. Returns $0$. \\

\begin{displaymath}
\myepsilon := \left[
   \begin{array}{c}
       \epsilon_{xx} \\
       \epsilon_{yy}   \\
       2 \gamma_{xy}   
   \end{array} 
 \right]
\end{displaymath}

{\em const Vector \&getStress (void); } 

Returns the material stress vector, $\mysigma$, for the current
trial strain. \\

\begin{displaymath}
\mysigma := \left[
   \begin{array}{c}
       \sigma_{xx} \\
       \sigma_{yy}   \\
       \tau_{xy}   
   \end{array} 
 \right]
\end{displaymath}

{\em const Matrix \&getTangent (void); } 

Returns the material tangent stiffness matrix, $\D$. \\

\begin{displaymath}
\D := \frac{E}{(1+\nu)(1-2\nu)} \left[
   \begin{array}{ccc}
         1-\nu &     \nu &      0 \\
           \nu &   1-\nu &      0 \\
             0 &       0 & 1-2\nu
   \end{array} 
 \right]
\end{displaymath}

{\em int commitState (void); } 

Returns $0$. \\

{\em int revertToLastCommit (void); } 

Returns $0$. \\

{\em int revertToStart (void); } 

Returns $0$. \\

{\em NDMaterial *getCopy (void); } 

Returns a pointer to a new ElasticIsotropicPlaneStrain2D, with the
same values for {\em tag}, {\em E}, and $\nu$. It is up to the
caller to ensure that the destructor is invoked. \\
