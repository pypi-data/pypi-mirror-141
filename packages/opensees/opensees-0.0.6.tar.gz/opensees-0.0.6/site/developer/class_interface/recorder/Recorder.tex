%File: ~/recorder/Recorder.tex
%What: "@(#) Recorder.tex, revA"

\noindent {\bf Files}   \\
\#include $<\tilde{ }$/recorder/Recorder.h$>$  


\noindent {\bf Class Declaration}  \\
class Recorder 


\noindent {\bf Class Hierarchy} \\
{\bf Recorder} 


\noindent {\bf Description}  \\
\indent The Recorder class is an abstract class which is introduced to allow
information to be saved during the analysis. The interface defines two
pure virtual methods {\em record()} and {\em playback()}. {\em
record()} is a method which is called by the Domain object during a
{\em commit()}. The {\em playback()} method can be called by the analyst after
the analysis has been performed. \\

\noindent {\bf Class Interface} \\
// Constructor 

{\em Recorder();}\\ 

// Destructor 

{\em virtual $\tilde{ }$Recorder();}\\ 

// Public Methods  

{\em virtual int record(int commitTag) =0;}

{\em virtual int playback(int commitTag) =0;}

\indent {\em virtual void restart(void) =0;}\\ 

\noindent {\bf Constructor}  \\
\indent {\em Recorder();}  \\ 
Does nothing.\\

\noindent {\bf Destructor} \\
\indent {\em virtual~ $\tilde{}$Recorder();}\\ 
Does nothing. \\

\noindent {\bf Public Methods }  \\
{\em virtual int record(int commitTag) =0;}

Invoked by the Domain object after {\em commit()} has been invoked on all the
domain component objects. What the Recorder records depends on the
concrete subtype. \\

{\em virtual int playback(int commitTag) =0;}

Invoked by the analyst after the analysis has been performed. What the
Recorder does depends on the concrete subtype. \\

\indent {\em virtual void restart(void) =0;}\\ 
Invoked by the Domain object when {\em revertToStart()} is invoked on
the Domain object. What the Recorder does depends on the concrete subtype. 